\chapter{序論}
\section{研究背景}
\label{sec:background}

近年、情報通信量の爆発的な増加に伴い、データセンターなどにおける消費電力の増大が喫緊の課題となっている。この課題を解決する技術として、低消費電力かつ小型化が可能なシリコンフォトニクス技術を用いたフォトニック集積回路(PIC: Photonic Integrated Circuit)が注目されている。PICは、光信号の伝送、波長分割多重、スイッチング、モニタリングなど多岐にわたる機能を担い、回路規模は年々大規模化・複雑化の傾向にある。

PIC設計において、電磁界シミュレーションは必須の工程であるが、回路規模の拡大に伴い、シミュレーションに必要な計算資源(メモリや計算時間)の確保が困難になっている。特に、有限差分時間領域(FDTD)法などの高精度なシミュレーション手法では、微細なメッシュ設定が求められ、計算コストが指数関数的に増大する。過去の研究においても、計算時間の削減を目的とした超解像技術の適用などが試みられているが\cite{toyota_cleopr_2024}、抜本的な設計効率の向上が求められている。

設計者は通常、共振波長や透過率などの特定の物理量を確認した後、シミュレーションで得られた詳細な電磁界分布データ($\text{E}_x, \text{E}_y, \text{E}_z$などの数値データ)を、そのデータ容量の大きさから破棄してしまうことが多い。一方で、結果の可視化のために生成される電磁界画像や強度画像はデータ容量が比較的小さく、設計過程の記録として保持されている場合がある。これらのアーカイブされた画像データを何らかの形で再利用できれば、シミュレーションのやり直し(再計算)の必要性を低減し、設計プロセス全体の効率化に大きく貢献できると期待される。

\section{仮説}
\label{sec:hypothesis}

電磁界シミュレーションから得られる電磁界分布画像(位相情報を含む)と、その二乗に比例する強度分布画像(光のパワーを示す)の間には、物理法則に基づいた一意な対応関係が存在する。この対応関係は、設計者の経験や直感によって推測可能である。

本研究では、この複雑な非線形な対応関係を、畳み込みニューラルネットワーク(CNN: Convolutional Neural Network)を用いて学習できると仮定する。具体的には、あるシミュレーション条件(例:Cバンド)で得られた電磁界画像と強度画像のペアを学習データとして用いることで、CNNモデルは電磁界画像から強度画像を忠実に再構成する能力を獲得できる。

さらに、学習に用いた波長帯域(例:Cバンド)とは異なる波長帯域(例:Oバンド)や、異なる視覚化のためのカラーマップで得られた電磁界画像をモデルに入力しても、モデルは両者の間の基本的な物理的関係性を捉えているため、強度画像を正確に推定できる汎用性を持つと仮説を立てる。

\section{研究目的}
\label{sec:purpose}

本研究の目的は、アーカイブされた電磁界シミュレーション画像データの再利用を可能にし、フォトニック集積回路(PIC)の設計効率を大幅に向上させる手法を確立することである。

具体的には、以下の二点を達成することを目指す。
\begin{enumerate}
    \item \textbf{電磁界画像から強度画像への再構成技術の開発:} 畳み込みニューラルネットワーク(CNN)を適用し、シミュレーションによって生成された電磁界分布画像を入力として、対応する強度分布画像を高速かつ高精度に再構成するモデルを開発する。
    \item \textbf{モデルの汎用性の実証:} 訓練データとは異なる波長帯域(例:Oバンド)や、視覚化のためのカラーマップで入力された電磁界画像に対しても、開発したモデルが30 dB以上のPSNR(Peak Signal-to-Noise Ratio)を維持し、視覚的に区別不可能なレベルで強度画像を再構成できることを定量的に実証する。
\end{enumerate}
