\chapter{序論}

\section{研究背景}
\label{sec:background}

近年、クラウドコンピューティングや5G通信の普及に伴い、
世界的に情報通信量が爆発的に増加している\cite{soumu_hakusho}(図\ref{fig:soumu_hakusyo})。
これに伴い、データセンターなどにおける情報処理機器の消費電力増大が課題となっている。
この課題を解決する技術として、低消費電力かつ大容量伝送が可能な
光集積回路(PIC: Photonic Integrated Circuit)が注目されている。
現在のPICは、光信号の伝送のみならず、波長分割多重、スイッチング、変調、受光、モニタリングなど
多岐にわたる機能をワンチップ上で実現しており、その回路規模は年々大規模化・複雑化の一途をたどっている。

\begin{figure}[htbp]
  \centering
  \includegraphics[width=0.8\linewidth]{figures/soumu_hakusyo.png}
  \caption{世界の通信トラフィックの推移予測\cite{soumu_hakusho}。今後も急激な増加が見込まれている。}
  \label{fig:soumu_hakusyo}
\end{figure}

PICの設計プロセスにおいて、電磁界シミュレーションはデバイスの特性を予測・最適化するために
不可欠な工程である。
しかし、回路の大規模化に伴い、シミュレーションに必要な計算資源(メモリや計算時間)の確保が
深刻な問題となっている。
PICを実現するために有望視されているプラットフォームシリコンフォトニクスがある。
シリコンをコアとし、その酸化膜をクラッドとするため、コアとクラッドの屈折率比を
大きくとれるため、PICの寸法を極限まで小さくすることができる。
しかしながら、そのシミュレーションでは、コアとクラッドの屈折率比が小さいことを
前提とするビーム伝搬法などを用いることができず、
Maxwell方程式を離散化して解く
時間領域有限差分
(FDTD: Finite-Difference Time-Domain)法を用いる必要がある。
FDTD法を用いて高精度な結果を得るためには、計算メッシュの微細化に伴い
計算コストは指数関数的に増大する(図\ref{fig:cal_time_graph})。

\begin{figure}[htbp]
  \centering
  \includegraphics[width=0.8\linewidth]{figures/fig_mesh_comparison.png}
  \caption{FDTDシミュレーションにおけるメッシュサイズと電磁界分布の解像度の関係。
微細なメッシュは高精度だが計算コストが高く、粗いメッシュは高速だが情報量が失われる。}
  \label{fig:mesh_comparison}
\end{figure}

\begin{figure}[htbp]
  \centering
  \includegraphics[width=0.8\linewidth]{figures/cal_time_graph.png}
  \caption{図\ref{fig:mesh_comparison}を計算したときにかかるメッシュサイズごとの計算時間。}
  \label{fig:cal_time_graph}
\end{figure}

この計算コスト増大の問題に対し、設計効率を維持・向上させるためには、
従来の設計手法の枠を超えた新たなアプローチが必要不可欠である。
特に近年、画像認識や生成の分野で著しい成果を上げている
深層学習(ディープラーニング)技術のPIC設計への応用が期待できる\cite{Wang:20}。

\section{研究の動機と着眼点}
\label{sec:motivation}

本研究では、PIC設計の効率化という課題に対し、機械学習を用いた画像処理技術のアプローチから、
二つの観点で解決を試みる。

第一のアプローチは、「シミュレーション自体の計算コスト削減」である。
FDTD法などのシミュレーションでは、粗いメッシュを用いれば計算時間は短縮されるが、
得られる電磁界分布の解像度が低下し、設計に必要な精度が得られないというトレードオフが
存在する(図\ref{fig:mesh_comparison}参照)。
ここで、低解像度の画像から高解像度の画像を推定する「超解像(Super-Resolution)」技術を
適用できれば、粗いメッシュによる高速な計算結果から、微細メッシュと同等の詳細な情報を
復元できる可能性がある。

第二のアプローチは、「過去のシミュレーションデータの再利用」である。
設計過程では膨大な数のシミュレーションが行われるが、
所望の設計パラメータを得たのちは、詳細な3次元電磁場分布といった
数値データは容量の都合で破棄されることが多く、データの再利用は難しい。この場合であっても
確認用の「可視化画像」(電磁界分布画像や強度分布画像)は比較的容量も小さく、
残されることが多い。これらの画像データには、デバイス内の光の振る舞いに関する重要な情報が含まれている。
もし、この情報が再利用可能になれば、再計算を行うことなく、
手元の画像アーカイブから必要な情報を即座に取得することが可能となり、
設計プロセスを大幅に効率化できる。
本研究では、これら「超解像」と「再利用」という二つのタスクを、
機械学習を用いた画像生成・変換問題として統一的に捉え、それぞれの有効性を検証する。

\section{仮説}
\label{sec:hypothesis}

本研究では、それぞれのタスクに対して以下の仮説を立てる。

\subsection*{超解像に関する仮説}
粗いメッシュで計算された低解像度の電磁界分布画像であっても、
光の波動としての基本的な振る舞いは保存されている。
CNNを用いることで、低解像度・高解像度ペアデータから、メッシュの粗さによって失われた情報を学習し、
低解像度のシミュレーション結果に対しても高精度な高解像度画像を再構成できると仮定する。

\subsection*{画像相互変換に関する仮説}
電磁界シミュレーションから得られる
電磁界分布画像と、強度分布画像の間には、
物理的な対応関係が存在する。図\ref{fig:field_vs_intensity}に示すように、
両者は見た目が大きく異なるが、熟練した設計者が視覚的に推測可能であるのと同様に、
機械学習によって推測可能であると仮定する。

さらに、ある波長帯域(例:Cバンド)で学習したモデルは、
異なる波長帯域(例:Oバンド)に対しても、一定の汎用性を持って変換が可能であると考える。

\begin{figure}[htbp]
  \centering
  \includegraphics[width=0.8\linewidth]{figures/fig_field_vs_intensity.png}
  \caption{電磁場画像(上)と対応する強度分布画像(下)の比較。
両者の間には物理的な対応関係が存在し、機械学習による相互変換が可能であると仮定する。}
  \label{fig:field_vs_intensity}
\end{figure}

\section{研究目的}
\label{sec:purpose}

本研究の目的は、機械学習を用いた画像処理技術を適用し、
以下の二つの技術を確立することで、PIC設計の効率化に貢献することである。

\begin{enumerate}
    \item \textbf{シミュレーション効率化のための超解像技術の確立}\\
粗いメッシュでの計算結果(低解像度画像)から、
微細メッシュ相当の電磁界分布(高解像度画像)を再構成する
CNN(Convolutional Neural Network)モデルを構築する。
PSNR(Peak Signal-to-Noise Ratio)のような定量的指標を用い、
計算時間の短縮が可能であることを実証する。

    \item \textbf{データ再利用のための画像相互変換技術の確立}\\
電磁界分布画像と強度分布画像を相互に変換するPix2Pixモデルを構築する。
視覚的に区別不可能な水準(PSNR 30 dB程度)で再構成が可能であることを示し、
過去の設計資産の有効活用による設計フロー効率化の可能性を提示する。
\end{enumerate}