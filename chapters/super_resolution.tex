\chapter{超解像によるシミュレーションの効率化}

\section{はじめに}
本章では、序論で述べた課題を解決するために、
機械学習を用いた超解像技術を適用し、
低解像度のシミュレーション結果から高解像度の電磁界分布を再構成する手法について述べる。
また、提案手法の有効性を検証するために行った、学習時と推論時で波長帯域を変えた結果
についても報告する。

\section{研究手順}
\label{sec:sr_methodology}

\subsection{シミュレーションデバイスとデータ生成}
機械学習の学習データとして電磁界パターンを収集するため、
光回路のFDTDシミュレーションを行った。

\subsubsection*{デバイス構造}
本研究では、干渉計ベースのシンプルな構造を採用した。
具体的には、2×2MMIから1つの入力導波路と2つの出力導波路を取り外したデバイスを
今回は使用している。(図\ref{fig:device_structure})。
この構造を用いることで、機械学習の学習素材となる電磁界パターンの画像を取得している。

\begin{figure}[htbp]
  \centering
  \includegraphics[width=0.8\linewidth]{figures/device_structure.png}
  \caption{本研究で使用した学習データ生成用のデバイス。}
  \label{fig:device_structure}
\end{figure}

\subsubsection*{計算条件とメッシュサイズ}
シミュレーションには3次元FDTD法を用いた。
以下のメッシュサイズを設定した。
\begin{itemize}
    \item \textbf{20 nmメッシュ}: 高解像度(High-Resolution: HR)。教師データとして使用。
    \item \textbf{40 nmメッシュ}: 低解像度(Low-Resolution: LR)。入力データとして使用。
\end{itemize}
メッシュサイズを細かくするほど計算時間は指数関数的に増大する。
20nmメッシュでの計算と比較して、40nmメッシュでは計算時間を短縮できるため、
この低解像度画像から高解像度画像を復元できれば設計効率は大きく向上する。

\subsection{SRCNNモデルと学習プロセス}
本研究では、超解像アルゴリズムとしてSRCNNを採用した。
SRCNNは、低解像度画像と高解像度画像のペアを学習させることで、
画像の特徴量を抽出し、低解像度画像から高解像度画像を再構成する。

\subsubsection*{データセットの構築}
シミュレーションで得られた電磁界分布画像を、それぞれ20枚の画像に分割した。
分割された画像のうち、電磁界がほとんど存在しない部分は、学習データとして無意味であり
ノイズとなる可能性があるため除外した。
残った画像ペアを以下のグループに分割した(図\ref{fig:train_data})。
\begin{figure}[htbp]
  \centering
  \includegraphics[width=0.8\linewidth]{figures/train_data.png}
  \caption{SRCNNの学習に使用したデータセット。}
  \label{fig:train_data}
\end{figure}

\subsubsection*{ネットワーク構造とパラメータ}
構築したSRCNNモデルの構造およびハイパーパラメータは以下の通りである。
\begin{itemize}
    \item \textbf{第1層(畳み込み層)}: フィルタ数 16, フィルタサイズ $3 \times 3$, 活性化関数 ReLU
    \item \textbf{第2層(畳み込み層)}: フィルタ数 3, フィルタサイズ $3 \times 3$, 活性化関数 ReLU
    \item \textbf{最適化手法}: Adam
    \item \textbf{損失関数}: 平均二乗誤差(MSE)
    \item \textbf{評価指標}: PSNR
    \item \textbf{エポック数}: 15000
    \item \textbf{バッチサイズ}: 32
\end{itemize}

\subsection{検証手順}
提案手法の性能と適用範囲を検証するため、以下の2つの検証を行った。

\begin{enumerate}
    \item \textbf{学習時と同じ波長帯のシミュレーション画像を超解像}\\
    Cバンド(波長1550nm帯)のシミュレーション結果を学習データとして用い、
    同帯域のシミュレーション画像の超解像を行う。
    
    \item \textbf{学習時とは異なる波長帯のシミュレーション画像を超解像}\\
    Cバンドのデータで学習したモデルを用いて、異なる波長帯域であるOバンド(波長1300 nm帯)
    のシミュレーション画像の超解像を行う。学習時とは異なる条件下でのモデルの汎用性を検証する。
\end{enumerate}

また、比較対象として、従来の画像補間手法であるバイリニア補間(Bilinear interpolation)
およびバイキュービック補間(Bicubic interpolation)による結果とも比較を行った。

\section{結果}
\label{sec:sr_results}

\subsection{Cバンド学習モデルの評価}
Cバンドのシミュレーション画像を学習データとして用いたモデルの評価を行った。
結果として、PSNRは劇的に向上し、約8dBの改善が見られた(図\ref{fig:C-band})。
再構成された画像のPSNRは30dBを超えており、視覚的にも高解像度画像と遜色のないレベルまで品質が向上した。
これにより、SRCNNの高い再構成能力が確認された。

\begin{figure}[htbp]
  \centering
  \includegraphics[width=1.0\linewidth]{figures/C-band.png}
  \caption{Cバンドを学習したモデルを用いてCバンド画像の超解像を行った結果。}
  \label{fig:C-band}
\end{figure}

\subsection{異波長(Oバンド)への適用結果}
Cバンドのデータで学習したモデルを用いて、Oバンドのシミュレーション画像の超解像を行った。
結果として、Cバンドでの推論時と同等の高いPSNR値が得られた(図\ref{fig:O-band})。
波長が異なっても、光の干渉や伝搬といった物理現象に由来する画像の特徴は共通しているため、
モデルがその特徴を捉えて汎化できたと考えられる。
この結果は、一度モデルを構築すれば、異なる波長条件のシミュレーションに対しても
再学習なしで適用できる可能性を示している。

\begin{figure}[htbp]
  \centering
  \includegraphics[width=1.0\linewidth]{figures/O-band.png}
  \caption{Cバンドを学習したモデルを用いてOバンド画像の超解像を行った結果。}
  \label{fig:O-band}
\end{figure}

\subsection{従来手法との比較}
SRCNNによる超解像と、従来の補間手法(バイリニア、バイキュービック)による結果を比較した。
図\ref{fig:method_comparison}に示すように、従来手法もある程度の画質改善効果はあるものの、
SRCNNと比較するとその効果は小さい。
SRCNNは、学習データから画像の特徴を学習することで、
単なる幾何学的な補間よりも高度な高解像度化を実現していると言える。

\begin{figure}[htbp]
  \centering
  \includegraphics[width=0.8\linewidth]{figures/method_comparison.png}
  \caption{SRCNNと従来の補間手法(バイリニア, バイキュービック)によるPSNRの比較。}
  \label{fig:method_comparison}
\end{figure}

\section{まとめ}
本章では、PIC設計における計算リソース削減を目的として、
SRCNNを用いたシミュレーション画像の超解像手法を提案・検証した。
検証の結果、SRCNNは従来の補間手法よりも優れた再構成能力を示した。
特に、学習時と同じ波長帯のものを超解像したとき、PSNR30dBを超える高精度な画像再構成が可能であることを確認した。
さらに、Cバンドで学習したモデルがOバンドのシミュレーション画像に対しても有効であることが示され、
提案手法の汎用性が実証された。
これらの結果は、SRCNNが将来の大規模かつ複雑なPIC設計において、
計算コストを削減するための重要な技術となり得ることを示している。