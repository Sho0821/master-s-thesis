\chapter{結論}
\label{chap:conclusion}

\section{本研究の総括}
本研究では、大規模化・複雑化するPIC
の設計プロセスにおける計算リソースの圧迫という課題に対し、
機械学習を用いた画像処理技術を適用することで、設計効率の向上を試みた。
具体的には、「低解像のシミュレーション結果から高解像のシミュレーション結果を再構成」と
「過去の画像資産の有効活用」という2つのアプローチを検討し、以下の成果を得た。

\subsection{超解像技術によるシミュレーションの効率化}
第3章では、SRCNNを用いて、粗いメッシュによる高速なFDTDシミュレーション結果
(低解像度画像)から、微細なメッシュ相当の電磁界分布(高解像度画像)を推定する手法を確立した。
検証の結果、提案モデルは従来の補間手法(バイリニア、バイキュービック)
を大きく上回る再構成精度を示し、PSNR 30dBを超え、LPIPSも0.1程度の高品質な画像を生成可能であることを確認した。
さらに、Cバンドのデータで学習したモデルがOバンドの未知の波長帯に対しても有効に機能することを実証し、
一度学習したモデルが幅広い設計条件下で汎用的に利用できる可能性を示した。
これにより、低解像の結果から高解像の結果を得て計算リソースを節約できる見通しを得た。

\subsection{画像相互変換による設計資産の再利用}
第4章では、数値データ破棄後も残存しやすい可視化画像に着目し、
Pix2Pixを用いた画像相互変換技術を開発した。
まず、電磁界分布画像から強度分布画像への変換において、
Pix2Pixが可視化カラーマップの設定に依存せず高い精度で物理的な対応関係を学習できることを示した。
さらに、Cバンドのデータで学習したモデルが、未知の波長帯であるOバンドを推論する
際も有効に機能することを実証した。

次に、位相情報が欠落している強度画像から電磁界分布を復元するという困難な逆問題に対し、
CAD画像(コア・クラッドの境界情報)および波長・励振点のラベルを補助入力とする
マルチ入力型Pix2Pixモデルを提案した。
この手法により、学習から除外した未知の波長条件(1550--1552 nm)に対しても、
干渉波を再現し、PSNR約30dBの再構成精度を達成した。

学習時に存在しなかった未知のデバイス形状に対する検証において、
知覚的評価指標であるLPIPSが0.1以下という極めて優れた値を記録した。
これにより、本モデルが単なる画素の変換を暗記しているのではなく、
デバイス構造と動作条件の相関から物理的にあるべき電磁界分布を高度に推論できていることを実証した。

また、強度画像が存在しない場合でも
CAD画像とラベル情報のみで物理則に基づいた予測が可能であることを示した。

これらの結果は、過去画像に蓄積されていた画像データが、再利用可能な情報源となることを意味している。

\section{本研究の意義}
本研究の意義は、機械学習を適用することによって、PIC設計における限られた計算リソースを節約可能であることを実証した点にある。
第一に、超解像技術の確立により、低解像度の計算結果から高解像度の分布を正確に再現できるようになった。これにより、本来であれば膨大な計算時間を要する「微細なメッシュでのシミュレーション」そのものを実行することなく、設計に必要な情報を得ることが可能となった。
第二に、画像相互変換技術を確立したことで、強度画像あるいは電磁界分布画像のいずれか一方が保存されていれば、他方の物理情報を画像から即座に取得・復元できることを検証した。これは、再シミュレーションの工数を排除できることを意味する。
構造情報から電磁界分布を直接推論できる能力は、設計初期段階における迅速な構造検討を可能にする物理シミュレータの代替としての価値を有する可能性がある。
以上の成果は、PIC設計における計算時間という課題を解消するものであり、過去の設計資産を情報源として再活用することで、光集積回路設計フローを効率化させるアプローチであると言える。