\chapter{参考技術}
\label{chap:reference}

\section{時間領域有限差分法 (FDTD法)}
\label{sec:fdtd}

時間領域有限差分法(Finite-Difference Time-Domain method: FDTD法)は、
マクスウェル方程式を空間および時間領域で差分化し、
電磁界の挙動を逐次計算する手法である。

真空中のマクスウェル方程式は以下の4式で表される。
\begin{align}
    \nabla \cdot \boldsymbol{D} &= \rho \label{eq:maxwell1} \\
    \nabla \cdot \boldsymbol{B} &= 0 \label{eq:maxwell2} \\
    \nabla \times \boldsymbol{E} + \frac{\partial \boldsymbol{B}}{\partial t} &= 0 \label{eq:maxwell3} \\
    \nabla \times \boldsymbol{H} - \frac{\partial \boldsymbol{D}}{\partial t} &= \boldsymbol{j} \label{eq:maxwell4}
\end{align}
ここで、$\boldsymbol{D}$は電束密度、$\boldsymbol{E}$は電界、$\boldsymbol{B}$は磁束密度、$\boldsymbol{H}$は磁界、$\rho$は電荷密度、$\boldsymbol{j}$は電流密度を表す。

FDTD法では、Yee格子と呼ばれるスタガード格子を用い、
電界と磁界を空間的・時間的に半ステップずらして配置することで計算を行う。

\section{Multi-Mode Interferometer (MMI)}
\label{sec:mmi}

光集積回路において、光の分岐・合波やスイッチングを行う重要な素子
として多モード干渉計(Multi-Mode Interferometer: MMI)がある。
MMIは、多モード導波路内での高次モード間の干渉を利用して、
入力光の像を特定の位置に自己結像(Self-Imaging)させる素子である。
自己結像の原理により、多モード導波路の長さや幅を適切に設計することで、
1入力N出力の分岐や、N入力M出力の合波・分波機能を実現できる。
特に、$2 \times 2$ MMIカプラは、光通信の大容量化を支えるデジタルコヒーレント通信に
おいて重要な役割を果たしている。
例えば、四位相偏移変調(Quadrature Phase Shift Keying:QPSK)信号の復調器において、
信号光と局発光を干渉させ、90度の位相差を持つ成分を取り出すために用いられる。
MMIの出力における位相バランスや分岐比は、素子の幾何学的寸法に敏感であるため、
設計段階における高精度なシミュレーションが不可欠である。

\section{畳み込みニューラルネットワーク (CNN)}
\label{sec:cnn}

\subsection{概要}
畳み込みニューラルネットワーク(Convolutional Neural Network: CNN)は、
画像認識や画像生成の分野で広く用いられる深層学習モデルの一種である。
画像内の局所的な特徴を効率的に抽出し、位置ずれに対する不変性を持つことから、
光回路の電磁界分布のような空間的なパターンを持つデータの処理に適している。

\subsection{各層の役割}
\begin{itemize}
    \item \textbf{畳み込み層}: 
    入力画像に対してカーネル(フィルタ)と呼ばれる重み行列をスライドさせながら積和演算を行い、
    特徴マップを生成する。これにより、電磁界の干渉縞などの局所的特徴を抽出する。
    \item \textbf{プーリング層}: 
    特徴マップを縮小(ダウンサンプリング)し、計算量の削減や過学習の抑制を行う。
    \item \textbf{全結合層}: 
    抽出された特徴量を統合し、最終的な出力を行う。
\end{itemize}

\section{超解像技術}
\label{sec:super_resolution}

超解像(Super-Resolution: SR)とは、低解像度の画像から高解像度の画像を推定・生成する技術である。
本研究における超解像タスクでは、粗いメッシュで計算された低解像度な電磁界分布画像を入力とし、
微細なメッシュで計算された高解像度な電磁界分布画像を推定することを目指す。
SRCNN (Super-Resolution CNN) \cite{dong2014}は、
低解像度画像から高解像度画像への非線形なマッピングを学習することで、
従来の補間手法(バイキュービック法など)よりも鮮明な画像を再構成することを可能にした。
本研究でもこの考え方を応用し、計算コストの削減を図る。

\section{画像相互変換技術}
\label{sec:image_conversion}

画像相互変換は、ある画像を別の画像へ変換する技術である。
光回路シミュレーションにおいては、電磁界分布と強度分布という
異なる物理量の可視化画像が存在する。
これらは物理法則によって紐づいているが、
その関係は非線形である。機械学習を用いることで、
この複雑な対応関係を学習し、一方の画像から他方の画像を再構成することを試みる。

\subsection{Pix2Pix}
Pix2Pixは、条件付き敵対的生成ネットワーク(Conditional Generative Adversarial Networks: cGAN)
に基づいた画像対画像の変換モデルである\cite{isola2017}。
Pix2Pixは、入力画像から目的の画像を生成する生成器(Generator)と、
生成された画像が本物か偽物かを判別する識別器(Discriminator)の2つのネットワークで構成される。
生成器は識別器を騙すように、識別器は生成器の画像を正しく見破るように学習を進めることで、
生成器はより本物に近い高精度な画像を生成できるようになる。
本研究における画像変換では、このPix2Pixのモデルを採用し、
電磁界分布画像と強度分布画像のペアデータを用いて学習を行うことで、
高度な変換則の獲得を目指す。

\section{評価指標 (PSNR)}
\label{sec:psnr}

画像の生成精度を定量的に評価する指標として、ピーク信号対雑音比(Peak Signal-to-Noise Ratio: PSNR)を用いる。
$m \times n$画素の正解画像$I$と生成画像$K$において、
平均二乗誤差(MSE)は以下で定義される。
\begin{equation}
    MSE = \frac{1}{mn}\sum_{i=0}^{m-1}\sum_{j=0}^{n-1}[I(i,j)-K(i,j)]^{2}
\end{equation}
画像の最大輝度値を$MAX_I$とすると、PSNRは以下の式で表される。
\begin{equation}
    PSNR = 10 \cdot \log_{10}\left(\frac{MAX_{I}^{2}}{MSE}\right) = 20 \cdot \log_{10}\left(\frac{MAX_{I}}{\sqrt{MSE}}\right)
\end{equation}
一般的に、PSNRが30 dB以上であれば、視覚的に劣化が気にならない高品質な画像であるとされる。\cite{huynh2008scope}