\chapter{参考技術}
\label{chap:reference}

本章では、本研究で用いる主要な技術および理論について述べる。まず、光回路の数値解析手法として、本研究の背景となるFDTD法およびデータ生成に用いるFMM法について解説する。次に、解析対象であるMMIカプラの原理について述べる。続いて、本研究の核となる機械学習技術として、畳み込みニューラルネットワーク(CNN)および、それを応用した超解像・画像再構成技術について説明する。最後に、解析結果の評価に用いる数値計算手法について触れる。

\section{光回路の数値解析手法}
\label{sec:numerical_analysis}

光集積回路の設計においては、マクスウェル方程式に基づいた厳密な電磁界シミュレーションが不可欠である。本研究に関連する主要な手法として、時間領域有限差分法(FDTD法)とフィルムモード整合法(FMM法)について述べる。

\subsection{時間領域有限差分法 (FDTD法)}
時間領域有限差分法(Finite-Difference Time-Domain method: FDTD法)は、マクスウェル方程式を空間および時間領域で差分化し、電磁界の挙動を逐次計算する手法である\cite{yee1966}。

真空中のマクスウェル方程式は以下の4式で表される。
\begin{align}
    \nabla \cdot \boldsymbol{D} &= \rho \label{eq:maxwell1} \\
    \nabla \cdot \boldsymbol{B} &= 0 \label{eq:maxwell2} \\
    \nabla \times \boldsymbol{E} + \frac{\partial \boldsymbol{B}}{\partial t} &= 0 \label{eq:maxwell3} \\
    \nabla \times \boldsymbol{H} - \frac{\partial \boldsymbol{D}}{\partial t} &= \boldsymbol{j} \label{eq:maxwell4}
\end{align}
ここで、$\boldsymbol{D}$は電束密度、$\boldsymbol{E}$は電界、$\boldsymbol{B}$は磁束密度、$\boldsymbol{H}$は磁界、$\rho$は電荷密度、$\boldsymbol{j}$は電流密度を表す。

FDTD法では、Yee格子と呼ばれるスタガード格子を用い、電界と磁界を空間的・時間的に半ステップずらして配置することで計算を行う。任意の形状や広帯域な波長特性を一度に解析できる利点がある一方で、精度を確保するためにメッシュサイズを波長の数十分の一以下にする必要があり、計算コスト(メモリと計算時間)が指数関数的に増大するという課題がある。

\subsection{フィルムモード整合法 (FMM法)}
フィルムモード整合法(Film Mode Matching method: FMM法)は、導波路を伝搬方向に対して均一な複数のセクションに分割し、各セクションにおける固有モードの重ね合わせとして電磁界を表現する手法である。
各セクションの界面において電磁界の接線成分が連続となるように境界条件を適用し、散乱行列(S行列)などを用いて全体の伝搬特性を解析する。FMM法は、MMIのような矩形構造の組み合わせで表現できるデバイスに対しては、FDTD法と比較して高速に計算が可能であるという特徴を持つ。本研究の一部では、学習データの効率的な生成に本手法が用いられている。

\section{多モード干渉計 (MMI)}
\label{sec:mmi}

光集積回路において、光の分岐・合波やスイッチングを行う重要な素子として多モード干渉計(Multi-Mode Interferometer: MMI)がある。

\subsection{動作原理と用途}
MMIは、多モード導波路内での高次モード間の干渉を利用して、入力光の像を特定のあいちに自己結像(Self-Imaging)させる素子である。自己結像の原理により、多モード導波路の長さや幅を適切に設計することで、1入力N出力の分岐や、N入力M出力の合波・分波機能を実現できる。

特に、$2 \times 2$ MMIカプラは、光通信の大容量化を支えるデジタルコヒーレント通信において重要な役割を果たしている。例えば、QPSK(Quadrature Phase Shift Keying:四位相偏移変調)信号の復調器において、信号光と局発光を干渉させ、90度の位相差を持つ成分を取り出すために用いられる。MMIの出力における位相バランスや分岐比は、素子の幾何学的寸法に敏感であるため、設計段階における高精度なシミュレーションが不可欠である。

\section{畳み込みニューラルネットワーク (CNN)}
\label{sec:cnn}

\subsection{概要}
畳み込みニューラルネットワーク(Convolutional Neural Network: CNN)は、画像認識や画像生成の分野で広く用いられる深層学習モデルの一種である。画像内の局所的な特徴(エッジ、テクスチャ、周期構造など)を効率的に抽出し、位置ずれに対する不変性を持つことから、光回路の電磁界分布のような空間的なパターンを持つデータの処理に適している。

\subsection{各層の役割}
\begin{itemize}
    \item \textbf{畳み込み層}: 入力画像に対してカーネル(フィルタ)と呼ばれる重み行列をスライドさせながら積和演算を行い、特徴マップを生成する。これにより、電磁界の干渉縞などの局所的特徴を抽出する。
    \item \textbf{プーリング層}: 特徴マップを縮小(ダウンサンプリング)し、計算量の削減や過学習の抑制を行う。
    \item \textbf{全結合層}: 抽出された特徴量を統合し、最終的な出力を行う。なお、画像の生成や変換を行うモデルでは、アップサンプリング層を用いて元の画像サイズへ復元する構成がとられることが多い。
\end{itemize}

\section{超解像および画像再構成技術}
\label{sec:super_resolution_reconstruction}

本研究では、CNNを応用した画像処理技術として、超解像と画像相互変換(再構成)の2つのアプローチに着目する。

\subsection{超解像技術 (Super-Resolution)}
超解像(Super-Resolution: SR)とは、低解像度の画像から高解像度の画像を推定・生成する技術である。Dongらによって提案されたSRCNN (Super-Resolution CNN) \cite{dong2014}は、低解像度画像から高解像度画像への非線形なマッピングを学習することで、従来の補間手法(バイキュービック法など)よりも鮮明な画像を再構成することを可能にした。
本研究における超解像タスクでは、粗いメッシュで計算された低解像度な電磁界分布画像を入力とし、微細なメッシュで計算された高解像度な電磁界分布画像を推定することを目指す。

\subsection{画像相互変換・再構成 (Image Conversion)}
画像相互変換は、あるドメインの画像を別のドメインの画像へ変換する技術である。光回路シミュレーションにおいては、電磁界分布(位相情報を含む干渉パターン)と光強度分布(パワーの流れ)という異なる物理量の可視化画像が存在する。
これらは物理法則(マクスウェル方程式)によって紐づいているが、その関係は非線形である。CNNを用いることで、この複雑な対応関係を学習し、一方の画像から他方の画像を再構成することが可能となる。これにより、保存されている可視化画像データからの物理情報の復元や、異なる可視化形式への変換が実現できる。

\subsection{Pix2Pix}
Pix2Pixは、Isolaらによって提案された、条件付き敵対的生成ネットワーク(Conditional Generative Adversarial Networks: cGAN)に基づいた画像対画像の変換(Image-to-Image Translation)モデルである\cite{isola2017}。
Pix2Pixは、入力画像から目的の画像を生成する生成器(Generator)と、生成された画像が本物か偽物かを判別する識別器(Discriminator)の2つのネットワークで構成される。
生成器は識別器を騙すように、識別器は生成器の画像を正しく見破るように学習を進める(敵対的学習)ことで、生成器はより本物に近い高精度な画像を生成できるようになる。
本研究における画像相互変換タスクでは、このPix2Pixのモデルを採用し、電磁界分布画像と強度分布画像のペアデータを用いて学習を行うことで、物理法則に基づいた高度な変換則の獲得を目指す。

\subsection{評価指標 (PSNR)}
画像の生成精度を定量的に評価する指標として、ピーク信号対雑音比(Peak Signal-to-Noise Ratio: PSNR)を用いる。$m \times n$画素の正解画像$I$と生成画像$K$において、平均二乗誤差(MSE)は以下で定義される。
\begin{equation}
    MSE = \frac{1}{mn}\sum_{i=0}^{m-1}\sum_{j=0}^{n-1}[I(i,j)-K(i,j)]^{2}
\end{equation}
画像の最大輝度値を$MAX_I$(通常は255)とすると、PSNRは以下の式で表される。
\begin{equation}
    PSNR = 10 \cdot \log_{10}\left(\frac{MAX_{I}^{2}}{MSE}\right) = 20 \cdot \log_{10}\left(\frac{MAX_{I}}{\sqrt{MSE}}\right)
\end{equation}
一般的に、PSNRが30 dB以上であれば、視覚的に劣化が気にならない高品質な画像であるとされる。

\section{数値計算手法}
\label{sec:curve_fitting}

\subsection{カーブフィッティング}
シミュレーション結果の画像から物理的な特性値(位相差など)を抽出するために、カーブフィッティング(曲線あてはめ)を用いる。本研究では、導波路内を伝搬する光の波形解析において、画像の画素値列(離散データ)から光の位相情報を推定するために利用する。
光波は正弦波としてモデル化できるため、振幅$A$、角周波数$\omega$、位相$\phi$、オフセット$C$をパラメータとする関数$f(x) = A \sin(\omega x + \phi) + C$へのフィッティング問題となる。

\subsection{Levenberg-Marquardt法}
非線形最小二乗法の解法として、Levenberg-Marquardt法(レーベンバーグ・マルカート法)を用いる。この手法は、最急降下法とガウス・ニュートン法を組み合わせたアルゴリズムであり、初期値依存性を低減しつつ安定して最適解を探索できる特徴がある。更新式は以下のように表される。
\begin{equation}
    \boldsymbol{x}_{k+1} = \boldsymbol{x}_{k} - (\boldsymbol{J}_{k}^{T}\boldsymbol{J}_{k} + \lambda \boldsymbol{I})^{-1}\boldsymbol{J}_{k}^{T}\boldsymbol{r}_{k}
\end{equation}
ここで、$\boldsymbol{x}$はパラメータベクトル、$\boldsymbol{J}$はヤコビ行列、$\boldsymbol{r}$は残差ベクトル、$\boldsymbol{I}$は単位行列、$\lambda$はダンピング係数である。
