\chapter{画像相互変換による設計資産の活用}
\label{chap:image_conversion}

\section{はじめに}
PICの設計プロセスにおいて、
電磁界シミュレーションはデバイス特性を予測するために不可欠である。
通常、設計者はシミュレーションを通じて位相差、分岐比などの値を確認する。
しかし、これらの値を算出するための元データとなる詳細な3次元電磁界分布データ($E_x, E_y, E_z, H_x, H_y, H_z$の6成分)
は、データ容量が極めて大きいため、ストレージ容量の制約から、
設計パラメータが確定した後には破棄されることが一般的である。

一方で、シミュレーション結果の確認や、論文・報告書への掲載のために生成された
可視化画像(電磁界分布画像や強度分布画像)は、比較的軽量に保存されており、
設計者の手元に蓄積されている場合が多い。
これらの画像データは数値データそのものではないが、
デバイス内の光の振る舞いに関する重要な情報を含んでいる。

本章では、これらの蓄積された画像データを設計資産として再利用し、
新たなシミュレーションを行うことなく必要な物理情報を復元する手法について述べる。
具体的には、機械学習を用いて、電磁界分布画像と強度分布画像の相互変換を試みる。

\section{電磁界分布画像から強度分布画像への変換}
\label{sec:field_to_intensity}

まず、第一段階として、情報量の多い電磁界分布画像から、
情報量の少ない強度分布画像を再構成する手法について述べる。
電磁界分布画像には位相情報が含まれているため、強度への変換は比較的容易であると予想される。
本研究では、この変換プロセスにPix2Pixを適用した。

\subsection{Pix2Pixによる画像変換の原理}
\label{sec:pix2pix_theory}
Pix2Pixは、cGANを用いた画像変換フレームワークである。

\subsubsection*{cGANの目的関数}
cGANでは、入力画像 $x$(ここでは強度分布画像)を条件として、
偽の画像を生成する生成器 $G$ と、入力 $x$ とターゲット $y$ のペアが本物か偽物かを判定する
識別器 $D$ が対抗的に学習を行う。
その目的関数 $\mathcal{L}_{cGAN}(G, D)$ は次式で表される。
\begin{equation}
    \mathcal{L}_{cGAN}(G, D) = \mathbb{E}_{x,y} [\log D(x, y)] + \mathbb{E}_{x,z} [\log(1 - D(x, G(x, z)))]
\end{equation}
ここで $z$ はノイズである。Pix2Pixの特徴は、
この敵対的損失に加え、生成画像が正解画像から大きく
乖離することを防ぐための $mathcal{L}_{L1}$ 損失を導入している点にある。
\begin{equation}
    \mathcal{L}_{L1}(G) = \mathbb{E}_{x,y,z} [\| y - G(x, z) \|_1]
\end{equation}
最終的な学習目標は、これらを組み合わせた次式の最適化となる。
\begin{equation}
    G^* = \arg \min_G \max_D \mathcal{L}_{cGAN}(G, D) + \lambda \mathcal{L}_{L1}(G)
\end{equation}
敵対的損失が高周波成分を、$mathcal{L}_{L1}$ 損失が低周波成分を担うことで、
物理的に整合性の取れた鮮明な電磁界分布の復元が可能となる。


\subsection{研究手順}
\label{sec:pix2pix_e_i}

\subsubsection*{シミュレーションデバイスとデータセットの生成}
学習データおよびテストデータを生成するために、シミュレーションを行った。

対象デバイスとして、SiコアとSiO$_2$クラッドからなる$2 \times 2$ MMIを採用した。
汎用性を検証するため、以下の2つの異なる波長帯域(CバンドおよびOバンド)で設計されたデバイスを用意した(図\ref{fig:device_structure2})。

\begin{itemize}
    \item \textbf{Cバンド用デバイス}: 入力ポート幅:440 nm 高さ:220 nm MMI領域のサイズ:2.21 $\mu$m $\times$ 16.5 $\mu$m。
    \item \textbf{Oバンド用デバイス}: 入力ポート幅:370 nm 高さ:220 nm MMI領域のサイズ:1.98 $\mu$m $\times$ 17.5 $\mu$m。
\end{itemize}

\begin{figure}[htbp]
  \centering
  \includegraphics[width=0.8\linewidth]{figures/device_structure2.png}
  \caption{本研究で使用した学習データ生成用のCバンド、Oバンドのデバイス。}
  \label{fig:device_structure2}
\end{figure}

本研究の目的で使用する電磁界分布画像と強度分布画像の例を図\ref{fig:electro_and_intensity}に示す。

\subsubsection*{可視化カラーマップの多様性}
設計現場において蓄積される画像資産は、
シミュレーションソフトの設定や報告書の意図に応じて、
多様なカラーマップで可視化されている場合がある。
提案手法が設計資産の再利用に貢献するためには、これらの表示形式の違いに依存せず、
一貫して強度分布を推定できる能力が求められる。そこで本研究では、
単一の表示形式だけでなく、複数の異なるカラーマップで描画された
電磁界分布画像を用意し、それらを入力とした際の変換精度を個別に検証した。
図\ref{fig:electro_and_intensity}に示したカラーマップの他に、図\ref{fig:other_e_field}に示した
カラーマップの電磁界分布画像を用意した。

\begin{figure}[htbp]
  \centering
  \includegraphics[width=0.8\linewidth]{figures/electro_and_intensity.png}
  \caption{電磁界分布画像と強度分布画像の例。}
  \label{fig:electro_and_intensity}
\end{figure}

\begin{figure}[htbp]
  \centering
  \includegraphics[width=0.8\linewidth]{figures/other_e_field.png}
  \caption{他のカラーマップの電磁界分布画像。}
  \label{fig:other_e_field}
\end{figure}

\subsubsection*{Pix2Pixモデルの学習}
電磁界分布画像を入力とし、強度分布画像を出力とするPix2Pixモデルを構築した。

\paragraph{データセットの構築}
波長1530nmから波長1549nmまでの範囲で1nm刻みでシミュレーションを行い、
20組の電磁界分布画像と強度分布画像のペアを取得した。
これらのペアをそれぞれのカラーマップごとに学習データとして使用した(図\ref{fig:image_pix2pix})。

\begin{figure}[htbp]
  \centering
  \includegraphics[width=0.8\linewidth]{figures/image_pix2pix.png}
  \caption{Pix2Pix学習のイメージ。}
  \label{fig:image_pix2pix}
\end{figure}

\paragraph{Pix2Pixモデルの構築}
本研究では、U-NetベースのGeneratorと、PatchGANベースのDiscriminatorを構築した。
各ネットワークの詳細は以下の通りである。
学習には、Cバンドのシミュレーションで得られた画像ペアのみを使用した。

\subsubsection*{Generatorの構成}
\begin{itemize}
    \item \textbf{構造}: U-Net(エンコーダ・デコーダ構造)
    \item \textbf{入力層}:
    \begin{itemize}
        \item 電磁界分布画像
    \end{itemize}
    \item \textbf{ダウンサンプリング層}: 7層
    \item \textbf{正規化・ドロップアウト}: InstanceNormalizationを採用。中間層においてDropout (0.5) を適用し過学習を抑制。
\end{itemize}

\subsubsection*{Discriminatorの構成}
\begin{itemize}
    \item \textbf{構造}: 70x70 PatchGAN
    \item \textbf{入力層}: 
    \begin{itemize}
        \item Generatorへの入力
        \item 生成画像または正解画像
    \end{itemize}
\end{itemize}

\subsubsection*{学習条件の詳細}
モデルの学習に使用したハイパーパラメータおよび条件は以下の通りである。

\begin{itemize}
    \item \textbf{最適化手法}: Adam ($\beta_1=0.5, \beta_2=0.999$)
    \item \textbf{学習率}: $2 \times 10^{-4}$
    \item \textbf{損失関数}: 
    \begin{itemize}
        \item 敵対的損失 (GAN Loss)
        \item L1損失 ($\lambda_{L1}=100.0$)
    \end{itemize}
    \item \textbf{エポック数}: 100
    \item \textbf{バッチサイズ}: 4
\end{itemize}

\subsubsection{検証手順}
提案手法の汎用性を検証するため、
学習完了後のモデルを用いて以下の2通りを推論した。

\begin{enumerate}
    \item \textbf{Cバンド推論}: 学習データと同じCバンドの電磁界分布画像を入力とする。本研究では波長1560nmでの電磁界分布画像を使用した。モデルによる学習の正当性を確認するものである。
    \item \textbf{Oバンド推論}: 学習データには含まれないOバンドの電磁界分布画像を入力とする。学習時とは異なる波長への適用可能性を検証するものである。
\end{enumerate}

\subsection{結果}

\subsubsection*{画像再構成の結果}
図\ref{fig:C-band_intensity}、図\ref{fig:color_map2_c-band}、図\ref{fig:color_map3_c-band}、
図\ref{fig:O-band_intensity}、図\ref{fig:color_map2_o-band}、図\ref{fig:color_map3_o-band}に、
電磁界分布画像から再構成された強度分布画像の結果を示す。
図\ref{fig:C-band_intensity}、図\ref{fig:color_map2_c-band}、図\ref{fig:color_map3_c-band}がCバンド(学習済み波長)、
図\ref{fig:O-band_intensity}、図\ref{fig:color_map2_o-band}、図\ref{fig:color_map3_o-band}がOバンド(未知の波長)の結果である。
いずれの場合も、電磁界分布画像から生成された再構成強度分布画像は、
中央の正解画像と視覚的におおむね一致している。
学習に使用していないOバンドのデータに対しても、
入力された電磁界の干渉パターンから強度分布を生成できており、
モデルが波長に依存しない対応関係を学習していることが示唆された。

\begin{figure}[htbp]
  \centering
  \includegraphics[width=0.8\linewidth]{figures/C-band_intensity.png}
  \caption{Pix2PixによるCバンド強度分布画像の再構成結果(カラーマップ1)。}
  \label{fig:C-band_intensity}
\end{figure}

\begin{figure}[htbp]
  \centering
  \includegraphics[width=0.8\linewidth]{figures/color_map2_c-band.png}
  \caption{Pix2PixによるCバンド強度分布画像の再構成結果(カラーマップ2)。}
  \label{fig:color_map2_c-band}
\end{figure}

\begin{figure}[htbp]
  \centering
  \includegraphics[width=0.8\linewidth]{figures/color_map3_c-band.png}
  \caption{Pix2PixによるCバンド強度分布画像の再構成結果(カラーマップ3)。}
  \label{fig:color_map3_c-band}
\end{figure}

\begin{figure}[htbp]
  \centering
  \includegraphics[width=0.8\linewidth]{figures/O-band_intensity.png}
  \caption{Pix2PixによるOバンド強度分布画像の再構成結果(カラーマップ1)。}
  \label{fig:O-band_intensity}
\end{figure}

\begin{figure}[htbp]
  \centering
  \includegraphics[width=0.8\linewidth]{figures/color_map2_o-band.png}
  \caption{Pix2PixによるOバンド強度分布画像の再構成結果(カラーマップ2)。}
  \label{fig:color_map2_o-band}
\end{figure}

\begin{figure}[htbp]
  \centering
  \includegraphics[width=0.8\linewidth]{figures/color_map3_o-band.png}
  \caption{Pix2PixによるOバンド強度分布画像の再構成結果(カラーマップ3)。}
  \label{fig:color_map3_o-band}
\end{figure}

\subsubsection*{定量評価}
再構成精度の定量的な評価として、PSNRおよびLPIPSを算出した。
結果を図\ref{fig:c_c_psnr}、図\ref{fig:c_o_psnr}、図\ref{fig:c_c_lpips}、図\ref{fig:c_o_lpips}に示す。

異なるカラーマップを用いて可視化された電磁界画像に対するPSNR、LPIPS評価の結果、
いずれのカラーマップにおいてもPSNR 30 dB程度、LPIPS 0.1程度の高精度な再構成が可能であることを
確認した。異なる色の割り当て規則を持つ入力に対しても、
適切な学習ペアを与えることで、モデルは色情報を解釈し、
強度分布へと再構成できることが実証された。

Cバンドでの推論結果は30 dBを超える高い値を示し(図\ref{fig:c_c_psnr})、
Oバンドでの推論結果においても約30 dBという値を達成した(図\ref{fig:c_o_psnr})\cite{cif_2025}。

LPIPSの数値はCバンドでの推論では0.1を下回り(図\ref{fig:c_c_lpips})、
Oバンドの推論では0.1に近い値を記録した(図\ref{fig:c_o_lpips})。

視覚的な比較結果からも明らかな通り、
干渉パターンの全体像やピークの位置、
導波路内での閉じ込めといった設計資産として再利用する上で
必要な物理的特徴を捉えられている。
PSNR、LPIPSが良好であることから、この結果は、ある波長帯域で構築した変換モデルが、
設計変更によって波長が変わったとしても、
再学習なしで一定の精度で適用可能であることを示している。

これらの結果から、多様の波長帯、カラーマップの過去の設計資産を活用できる可能性が示された。

\begin{figure}[htbp]
  \centering
  \includegraphics[width=0.8\linewidth]{figures/c_c_psnr.png}
  \caption{再構成された強度分布画像のPSNR評価結果。Cバンドの画像を学習したものでCバンドの画像を推論している。}
  \label{fig:c_c_psnr}
\end{figure}

\begin{figure}[htbp]
  \centering
  \includegraphics[width=0.8\linewidth]{figures/c_o_psnr.png}
  \caption{再構成された強度分布画像のPSNR評価結果。Cバンドの画像を学習したものでOバンドの画像を推論している。}
  \label{fig:c_o_psnr}
\end{figure}

\begin{figure}[htbp]
  \centering
  \includegraphics[width=0.8\linewidth]{figures/c_c_lpips.png}
  \caption{再構成された強度分布画像のLPIPS評価結果。Cバンドの画像を学習したものでCバンドの画像を推論している。}
  \label{fig:c_c_lpips}
\end{figure}

\begin{figure}[htbp]
  \centering
  \includegraphics[width=0.8\linewidth]{figures/c_o_lpips.png}
  \caption{再構成された強度分布画像のLPIPS評価結果。Cバンドの画像を学習したものでOバンドの画像を推論している。}
  \label{fig:c_o_lpips}
\end{figure}

\section{強度分布画像から電磁界分布画像への変換}
\label{sec:intensity_to_field}

次に、逆方向の変換、すなわち強度分布画像から電磁界分布画像への再構成について検討する。

\subsection{逆変換の困難性}
「電磁界$\to$強度」の変換と比較して、「強度$\to$電磁界」の変換は格段に難易度が高い。
これは、強度分布画像になった時点で位相情報が欠落しているためである。

本研究ではまず、前節で用いた標準的なPix2Pix構造を用いて強度から電磁界への逆変換を試行した。
しかし、図\ref{fig:failure_pix2pix}に示すように、生成された画像は全体的に不鮮明で、
電磁界分布において物理的に重要である鮮明な干渉縞が消失するという結果となった。
この課題を克服するため、本研究では入力画像、ラベル情報の追加を採用する。

\begin{figure}[htbp]
  \centering
  \includegraphics[width=0.8\linewidth]{figures/failure_pix2pix.png}
  \caption{電磁場画像の再構成失敗例。}
  \label{fig:failure_pix2pix}
\end{figure}

\subsection{研究手順}
\label{sec:pix2pix_i_e}

\subsubsection*{データセットの生成と構造バリエーション}
本タスクでは、デバイス構造、波長、励振点の組み合わせに
よる多様な干渉パターンを学習させるため、複数のデータセットを作成した。
学習に用いる各データには、物理条件を識別するための「ラベル情報」を付与した。
なお、本解析では、学習の効率化を目的として、
すべての電磁界分布画像、強度分布画像、をグレースケールとして処理した。
\ref{sec:field_to_intensity}の結果、
任意のカラーマップに対して変換を行えることを検証できたためである。

\begin{itemize}
    \item \textbf{構造バリエーション(CADパターン)}:
    \begin{itemize}
        \item \textbf{パターン1}: \ref{sec:field_to_intensity}でも使用したCバンド用MMIデバイス
        \item \textbf{パターン2}: 干渉計の縦幅を変更したモデル(2.25 $\mu$m)
        \item \textbf{パターン3}: 干渉計の横幅を変更したモデル(18.5 $\mu$m)
    \end{itemize}
    \item \textbf{波長割り当てルール(ラベル1)}:
    全帯域 $1530\text{--}1565\,\text{nm}$ において、以下の規則に従い波長を各構造に割り当て、これを入力ラベルとしてネットワークに与えた。
    \begin{itemize}
        \item $\lambda = 1530, 1533, 1536, \dots\,1563\text{nm}$ (パターン1に適用)
        \item $\lambda = 1531, 1534, 1537, \dots\,1564\text{nm}$ (パターン2に適用)
        \item $\lambda = 1532, 1535, 1538, \dots\,1565\text{nm}$ (パターン3に適用)
    \end{itemize}
    \item \textbf{励振位置の設定(ラベル2)}:
    入力導波路の左端からの距離として、$1\,\text{m}, 4\,\text{m}, 7\,\mu\text{m}$ の3パターンの励振点を設定し、これも入力ラベルとした。
\end{itemize}

本手法では、入力として強度分布画像に加え、コア・クラッドの境目を明示したCAD画像を同時に入力することで、
モデルが構造境界を認識し、適切な位相分布を再構成できるようにしている。
さらに、波長および励振点の数値を全画素に一定値として割り当て、入力する。
これにより、強度の流れだけでは決定できない位相をモデルに推論させる。

モデルの汎用性を検証するため、
1550nm,1551nm,1552nmの3つのデータについては学習から除外し、
未知の波長に対する推論用データとして確保した。

学習データの一例を図\ref{fig:pix2pix_samples}に示す。
これらは波長1530nm、励振位置1 $\mu$mの条件で得られた画像群である。
\clearpage

\begin{figure}[htbp]
  \centering
  \includegraphics[width=0.8\linewidth]{figures/pix2pix_samples.png}
  \caption{Pix2Pixに使用したデータセットの構成例(波長1530nm,励振位置1 $\mu$m,パターン1)。}
  \label{fig:pix2pix_samples}
\end{figure}

\subsubsection*{Pix2Pixモデルの構築}
\ref{sec:pix2pix_e_i}と同様に、U-NetベースのGeneratorと、PatchGANベースのDiscriminatorを構築した。各ネットワークの詳細は以下の通りである。

\paragraph{Generatorの構成}
\begin{itemize}
    \item \textbf{構造}: U-Net(エンコーダ・デコーダ構造)
    \item \textbf{入力層}: 
    \begin{itemize}
        \item 強度分布画像
        \item CAD構造画像:コアとクラッドの境界情報の保持
        \item 波長ラベル:全画素に正規化した波長値を割り当てたマップ
        \item 励振位置ラベル:全画素に正規化した励振位置を割り当てたマップ
    \end{itemize}
    \item \textbf{ダウンサンプリング層}: 7層
    \item \textbf{正規化・ドロップアウト}: InstanceNormalizationを採用。中間層においてDropout (0.5) を適用し過学習を抑制。
\end{itemize}

\paragraph{Discriminatorの構成}
\begin{itemize}
    \item \textbf{構造}: 70x70 PatchGAN
    \item \textbf{入力層}:
    \begin{itemize}
        \item Generatorへの入力
        \item 生成画像または正解画像
    \end{itemize}
\end{itemize}

\subsubsection*{学習条件の詳細}
モデルの学習に使用したハイパーパラメータおよび条件は以下の通りである。

\begin{itemize}
    \item \textbf{最適化手法}: Adam ($\beta_1=0.5, \beta_2=0.999$)
    \item \textbf{学習率}: $2 \times 10^{-4}$
    \item \textbf{損失関数}: 
    \begin{itemize}
        \item 敵対的損失 (GAN Loss)
        \item L1損失 ($\lambda_{L1}=100.0$)
    \end{itemize}
    \item \textbf{エポック数}: 1,000
    \item \textbf{バッチサイズ}: 4
\end{itemize}

\subsection{結果}
\label{sec:result_i_e}
電磁界分布画像の再構成結果を
図\ref{fig:pix2pix_1550nm}、図\ref{fig:pix2pix_1551nm}、図\ref{fig:pix2pix_1552nm}に示す。
入力データである強度分布からは位相情報は読み取れないが、
Pix2Pixによって生成された電磁界分布では、
正解データと極めて近い位置に干渉縞が鮮明に復元されていることがわかる。
CAD画像によってコアとクラッドの境界情報が与えられているため、
導波路構造に沿った分布が再構成されている。

\begin{figure}[htbp]
  \centering
  \includegraphics[width=0.8\linewidth]{figures/pix2pix_1550nm.png}
  \caption{Pix2Pixによる強度分布画像からの電磁界分布の再構成結果(波長1550nm)。}
  \label{fig:pix2pix_1550nm}
\end{figure}

\begin{figure}[htbp]
  \centering
  \includegraphics[width=0.8\linewidth]{figures/pix2pix_1551nm.png}
  \caption{Pix2Pixによる強度分布画像からの電磁界分布の再構成結果(波長1551nm)。}
  \label{fig:pix2pix_1551nm}
\end{figure}

\begin{figure}[htbp]
  \centering
  \includegraphics[width=0.8\linewidth]{figures/pix2pix_1552nm.png}
  \caption{Pix2Pixによる強度分布画像からの電磁界分布の再構成結果(波長1552nm)。}
  \label{fig:pix2pix_1552nm}
\end{figure}

定量評価として、テストデータに対するPSNR、LPIPSを
図\ref{fig:pix2pix_graph}、図\ref{fig:pix2pix_graph2}に示す。
検証の結果、PSNRが30 dBに達するケースが確認され、LPIPSも0.1を大きく下回る結果となった。
これは、提案モデルが強度分布画像、CADと波長・励振点の組み合わせから、
デバイス内部で発生すべき物理的な干渉パターンの型を学習できていることを意味する。
この結果は、CAD画像とラベルが物理的な枠組みを定義し、
強度分布画像が個別の構造における具体的な電磁界の振る舞い
を補完するという、入力構成の合理性を裏付けるものである。

\begin{figure}[htbp]
  \centering
  \includegraphics[width=0.8\linewidth]{figures/pix2pix_graph.png}
  \caption{テストデータ(波長1550nm,1551nm,1552nm)に対するPSNR評価結果。}
  \label{fig:pix2pix_graph}
\end{figure}

\begin{figure}[htbp]
  \centering
  \includegraphics[width=0.8\linewidth]{figures/pix2pix_graph_lpips.png}
  \caption{テストデータ(波長1550nm,1551nm,1552nm)に対するLPIPS評価結果。}
  \label{fig:pix2pix_graph_lpips}
\end{figure}

\subsection{未知のデバイス構造に対する汎用性検証}
\label{sec:unseen_geometry}
モデルの汎用性を確認するために、
学習データに一切含まれていなかった形状を持つデバイスに対する推論実験をした。

\paragraph{検証条件}
対象とした構造は、干渉計が縦 $2.25\,\mu\text{m} \times$ 横 $18.5\,\mu\text{m}$ のMMIである。
学習データには「縦 $2.25\,\mu\text{m}$(パターン2)」や「横 $18.5\,\mu\text{m}$(パターン3)」のバリエーションは含まれていたが、
この寸法は初見となる。このデバイスに対し、波長 1530 nm、励振点 $1, 4, 7\,\mu\text{m}$ の条件で推論を行った。

\paragraph{結果}
図\ref{fig:unseen_inference}に、正解データと推論結果の比較を示す。
導波路境界における電磁界の閉じ込めや、多モード干渉特有の複雑な波面が鮮明に再現されていることがわかる。
定量評価の結果を図\ref{fig:psnr_graph},図\ref{fig:lpips_graph}に示す。
検証の結果、この未知構造に対してはPSNRが30 dBを大きく下回る結果となった。
これは、未学習のデバイス寸法に起因する全体的な輝度バランスの変動や
コントラストの変化が、画素ごとの誤差を平均化するPSNRの数値に敏感に反映されたためと考えられる。
一方、LPIPSは0.1以下を記録しており、\ref{sec:result_i_e}には及ばないが、
知覚的に極めて類似した干渉縞が生成されていることが確認された。
未知のデバイス構造に対しての適用ゆえに、干渉縞を十分に再構成できていないため、
\ref{sec:result_i_e}と比較してPSNR、LPIPSともに値は劣っている。
だが、学習データにない形状に対してもLPIPS 0.1以下という再現性を維持している点は、
本手法が未知の設計パラメータに対しても機能する汎用性を備えていることを実証している。

\begin{figure}[htbp]
  \centering
  \includegraphics[width=0.8\linewidth]{figures/unseen_inference.png}
  \caption{未知の構造における電磁界分布画像の再構成結果。}
  \label{fig:unseen_inference}
\end{figure}

\begin{figure}[htbp]
  \centering
  \includegraphics[width=0.8\linewidth]{figures/psnr_graph.png}
  \caption{未知の構造における再構成電磁界分布画像に対するPSNR評価結果。}
  \label{fig:psnr_graph}
\end{figure}

\begin{figure}[htbp]
  \centering
  \includegraphics[width=0.8\linewidth]{figures/lpips_graph.png}
  \caption{未知の構造における再構成電磁界分布画像に対するLPIPS評価結果。}
  \label{fig:lpips_graph}
\end{figure}

\subsection{強度分布画像が無い場合の電磁界分布画像再構成の検証}
強度分布画像が入力に無い場合の推論可能性を検証するため、強度画像を除外し、
CAD画像とラベル情報のみから電磁界分布を推論する比較実験をした。
構造バリエーション、、波長割り当てルール、励振点の設定、Pix2Pixモデルの構成は\ref{sec:pix2pix_i_e}と同様のものを使用した。
強度画像を除外したときのPix2Pixの学習の様子を図\ref{fig:i_e_no_intensity}に示す。
テストデータとして使用するのは\ref{sec:pix2pix_i_e}と同様に、1550nm,1551nm,1552nmの3つである。

\begin{figure}[htbp]
  \centering
  \includegraphics[width=0.8\linewidth]{figures/i_e_no_intensity.png}
  \caption{強度画像無し学習の図。}
  \label{fig:i_e_no_intensity}
\end{figure}

\subsubsection*{結果}
図\ref{fig:pix2pix_graph2}および図\ref{fig:pix2pix_graph2_lpips}に、
強度画像無し(CAD + ラベルのみ)の場合の評価結果を示す。
強度画像を入力に含まない場合であっても、\ref{sec:pix2pix_e_i}の結果には及ばないものの、
PSNRが30 dBに達するケースが確認され、LPIPSも0.1を下回る結果となった。
これは、提案モデルが単に入力画像の画素値を変換しているのではなく、
導波路構造(CAD)と波長・励振位置の組み合わせのみから、
デバイス内の物理的な干渉パターンを学習できていることを意味する。
つまり、モデルが簡易的な物理シミュレータとして働いており、
強度画像がない状態でも、物理法則に則った妥当な分布を推論
できるポテンシャルを備えていることが明らかとなった。
強度画像を追加することで復元はより安定かつ高精度になるが、
構造情報のみからの推論能力は、設計の効率化において有用な知見である。

\begin{figure}[htbp]
  \centering
  \includegraphics[width=0.8\linewidth]{figures/pix2pix_graph2.png}
  \caption{テストデータ(波長1550nm,1551nm,1552nm)に対するPSNR評価結果(入力強度画像無し)。}
  \label{fig:pix2pix_graph2}
\end{figure}

\begin{figure}[htbp]
  \centering
  \includegraphics[width=0.8\linewidth]{figures/pix2pix_graph2_lpips.png}
  \caption{テストデータ(波長1550nm,1551nm,1552nm)に対するLPIPS評価結果(入力強度画像無し)。}
  \label{fig:pix2pix_graph2_lpips}
\end{figure}


\section{まとめ}
本章では、数値シミュレーションデータの廃棄後に残存する画像資産を再利用するための画像相互変換技術について検討した。
電磁界から強度への順変換だけでなく、困難な強度から電磁界への逆変換を実現した。
特に、CAD画像による構造境界の制約と、波長・励振点ラベルによる
物理条件をネットワークに統合して入力する手法を提案した。
検証の結果、強度分布から電磁界分布画像への変換はカラーマップに依存せず実現でき、
逆変換はPix2Pixに物理的補助情報を統合することで、PSNR 約30 dB、LPIPS 0.1を下回る復元が可能であることを実証した。
学習時に存在しない未知のデバイス形状に対する検証においては、LPIPSにおいて極めて優れた値を記録し、モデルが物理則を高度に学習できていることが確認された。
また、強度分布画像がない場合でも物理則に基づいた予測が可能であるという知見が得られた。
これらにより機械学習によるPIC設計支援の新たな可能性が示された。